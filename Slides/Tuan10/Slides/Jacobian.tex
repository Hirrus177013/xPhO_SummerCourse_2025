\begin{frame}{Ma trận Jacobian và liên hệ vận tốc giữa các tọa độ}
    \begin{columns}
        \column{0.5\textwidth}
            \begin{itemize}
                \item Cho hai hệ tọa độ: \( q = \left[ q_1, q_2, q_3 \right]^T\) và \( P = \left[ x, y, z \right]^T\).
                \item Dựa vào phép đạo hàm toàn phần
            \end{itemize}
            \begin{align}
                \dot{x} &= \dfrac{\partial x}{\partial q_1} \dot{q}_1 + \dfrac{\partial x}{\partial q_2} \dot{q}_2 + \dfrac{\partial x}{\partial q_3} \dot{q}_3, \\
                \dot{y} &= \dfrac{\partial y}{\partial q_1} \dot{q}_1 + \dfrac{\partial y}{\partial q_2} \dot{q}_2 + \dfrac{\partial y}{\partial q_3} \dot{q}_3, \\
                \dot{x} &= \dfrac{\partial x}{\partial q_1} \dot{q}_1 + \dfrac{\partial z}{\partial q_2} \dot{z}_2 + \dfrac{\partial z}{\partial q_3} \dot{q}_3.
            \end{align}
        \column{0.5\textwidth}
            \begin{itemize}
                \item Ma trận Jacobian
            \end{itemize}
            \begin{equation}
                J = \left[ \begin{array}{ccc}
                    \dfrac{\partial x}{\partial q_1} & \dfrac{\partial x}{\partial q_2} & \dfrac{\partial x}{\partial q_3} \\
                    \dfrac{\partial y}{\partial q_1} & \dfrac{\partial y}{\partial q_2} & \dfrac{\partial y}{\partial q_3} \\
                    \dfrac{\partial x}{\partial q_1} & \dfrac{\partial z}{\partial q_2} & \dfrac{\partial z}{\partial q_3}
                \end{array} \right].
            \end{equation}
            \begin{itemize}
                \item Ứng dụng: \(\dot{P} = J \dot{q}\).
            \end{itemize}
    \end{columns}
\end{frame}

\begin{frame}{Ma trận Jacobian và liên hệ vận tốc giữa các tọa độ}
    \begin{itemize}
        \item Ứng dụng cho Robot Scara.
    \end{itemize}
            \begin{align}
                \dot{x} &= \left[ -a_1 \sin \left( q_1 \right) - a_2 \sin \left( q_1 + q_2 \right) \right] \dot{q}_1 - a_2 \sin \left( q_1 + q_2 \right) \dot{q}_2, \\
                \dot{y} &= \left[ a_1 \cos \left( q_1 \right) + a_2 \cos \left( q_1 + q_2 \right) \right] \dot{q}_1 - a_2 \cos \left( q_1 + q_2 \right) \dot{q}_2, \\
                \dot{z} &= -\dot{q}_3.
            \end{align}
    \begin{itemize}
        \item Tính độ lớn vận tốc
    \end{itemize}
    \begin{equation}
        v^2 = \dot{x}^2 + \dot{y}^2 + \dot{z}^2
        = \left[ \begin{array}{ccc}
            \dot{x} & \dot{y} & \dot{z}
        \end{array} \right]
        \left[ \begin{array}{c}
            \dot{x} \\
            \dot{y} \\
            \dot{z}
        \end{array} \right]
        = \dot{P}^T \dot{P}
        = \dot{q}^T \left( J^T J \right) \dot{q}.
    \end{equation}
\end{frame}