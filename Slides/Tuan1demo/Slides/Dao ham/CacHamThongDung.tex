\begin{frame}{Đạo hàm của các hàm thông dụng}
    Đạo hàm của hàm đa thức:
    \begin{tcolorbox}[colback=blue!10, colframe=blue!50!black, title=Quy tắc lũy thừa]
    Nếu $n$ là số thực tùy ý, thì
    \begin{equation}
        \dfrac{d}{dx}=nx^{n-1}
    \end{equation}
    \end{tcolorbox}
\end{frame}
\begin{frame}{Đạo hàm của các hàm thông dụng}
    Đạo hàm của các hàm lượng giác:
    \begin{tcolorbox}[colback=blue!10, colframe=blue!50!black, title=]
    \begin{columns}
        \column{0.3\textwidth}
        $$
        \begin{aligned}
            &\dfrac{d}{dx}(\sin x)=\cos x\\
            &\dfrac{d}{dx}(\cos x)=-\sin x\\
            &\dfrac{d}{dx}(\tan x)=\sec^2 x
        \end{aligned}
        $$
    \column{0.3\textwidth}
    $$
    \begin{aligned}
        &\dfrac{d}{dx}(\csc x)=-\csc x \cot x\\
        &\dfrac{d}{dx}(\sec x)=\sec x \tan x\\
        &\dfrac{d}{dx}(\cot x)=-\csc^2 x
        \end{aligned}
    $$
    \end{columns}
    \end{tcolorbox}
\end{frame}
\begin{frame}{Đạo hàm của các hàm thông dụng}
    Đạo hàm hàm mũ
    \begin{tcolorbox}[colback=blue!10, colframe=blue!50!black, title=]
    \begin{equation}
    \begin{aligned}
        &\dfrac{d}{dx}a^x=a^x \ln a\\
        &\frac{d}{dx}\log_ax=\dfrac{1}{x\ln a}
        \end{aligned}
    \end{equation}
    \end{tcolorbox}
\end{frame}