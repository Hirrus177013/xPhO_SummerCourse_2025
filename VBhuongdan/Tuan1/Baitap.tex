\subsection*{Tập hợp xác định và giá trị của hàm số}\label{bai1.1-1.2}
\textbf{Bài 1.1:} Tìm miền xác định của các hàm số sau
\begin{enumerate}[label=(\alph*)]
\item \(\frac{x-2}{2x-1}\)
\item \(\frac{ln(1+x)}{x-1}\)
\item \(\sqrt{1-2x}+3\arcsin\frac{3x-1}{2} (\sin{x}=y\leftrightarrow x=\arcsin y) \)
\item \(\frac{1}{xe^x}\)
\item \(\ln{(3x+1)}+2\ln{(x+1)}\)
\end{enumerate}

\vspace{5pt}
\textbf{Bài 1.2:} Tìm tập hợp giá trị của các hàm số sau

\begin{enumerate}[label=(\alph*)]
\item \(x^2 -6x+5\)
\item \(2=3\sin x\)
\item \(\lvert x\rvert +x+1=y+\lvert y\rvert\)
\item \(4^{-x^2}\)
\end{enumerate}


\subsection*{Vẽ đồ thị của hàm}
\emph{Chú ý}: Ta có thể vẽ đồ thị của hàm có dạng $y=Af(k(x-a))+b$ theo đồ thị của hàm $f(x)$
\begin{itemize}
    \item $y=f(x-a)$: đồ thị ban đầu được tịnh tiến theo trục $Ox$ một đại lượng $a$.
    \item $y=f(x)+b$: đồ thị ban đầu được tịnh tiến theo trục $Oy$ một đại lượng $b$.
    \item $y=Af(x)$: đồ thị xuất phát được giãn ra $A$ lần theo trục $Oy$.
    \item $y=f(kx)$: đồ thị xuất phát được giãn ra $1/k$ lần theo trục $Ox$.
\end{itemize}
\textbf{Bài 1.3:} Vẽ đồ thị các hàm số trong hai bài tập ở trên bằng
a. Desmos \newline
b. Python (đối với hàm tuần hoàn thì vẽ trong khoảng $[-\pi;\pi]$; đối với các hàm khác, lựa chọn điểm đầu và cuối sao cho thu được mọi miền của hàm)

\textbf{Bài 1.4}: Vẽ một hình tam/tứ/ngũ/lục giác đều bằng Desmos và Python.

\subsection*{Hàm hợp}
\textbf{Bài 1.5:}Các hàm số trong phần \ref{bai1.1-1.2} là hàm hợp của những hàm nào? Hãy phân tích cụ thể thứ tự của chúng.
\textbf{Bài 1.6\(\star\):} Nguyên lý quy nạp\newline
Cho $S_n$ là một phát biểu về số nguyên dương $n$. Giả sử rằng:
\begin{itemize}
    \item $S_1$ đúng.
    \item $S_{k+1}$ đúng khi $S_k$ đúng.
\end{itemize}
Khi đó $S_n$ đúng với tất cả các số nguyên dương $n$.\newline
Sử dụng điều này để giải các bài toán sau:
\begin{enumerate}[label=(\alph*)]
\item Nếu $f_0(x) =x/(x+1)$ và $f_{n+1}(x)=f_0(f_n(x))$ với $n=0, 1, 2,\dots$, tìm một công thức cho $f_n(x)$.
\item Nếu $f_0(x) =x^2$ và $f_{n+1}(x)=f_0(f_n(x))$ với $n=0, 1, 2,\dots$, tìm một công thức cho $f_n(x)$.
\end{enumerate}




