\subsection*{Hàm số}
\textbf{Bài 1.1:} Tìm miền xác định của các hàm số sau
\begin{enumerate}[label=(\alph*)]
<<<<<<< HEAD
\item $\frac{\ln(1+x)}{x-1}$
\item $\sqrt{1-2x}+3\arcsin\left(\frac{3x-1}{2}\right)$ \quad ($\sin{x}=y\leftrightarrow x=\arcsin y$)
\item $\frac{1}{x e^x}$
\item $\ln{(3x+1)}+2\ln{(x+1)}$
=======
\item \(\frac{x-2}{2x-1}\)
\item \(\frac{ln(1+x)}{x-1}\)
\item \(\sqrt{1-2x}+3\arcsin\frac{3x-1}{2} (\sin{x}=y\leftrightarrow x=\arcsin y) \)
\item \(\frac{1}{xe^x}\)
\item \(\ln{(3x+1)}+2\ln{(x+1)}\)
>>>>>>> 034e26e813f325d967f66d2871efe4ac7d54c9a0
\end{enumerate}

\vspace{5pt}
\textbf{Bài 1.2:} Tìm tập hợp giá trị của các hàm số sau
\begin{enumerate}[label=(\alph*)]
\item $x^2 -6x+5$
\item $2=3\sin x$
\item $|x| + x + 1 = y + |y|$
\item $4^{-x^2}$
\end{enumerate}
\vspace{5pt}

\textbf{Bài 1.3:} Chứng minh
\begin{enumerate}[label=(\alph*)]
\item $\cos(\alpha+\beta)=\cos\alpha\cos\beta - \sin\alpha\sin\beta.$
\item $\sin(\alpha+\beta)=\sin\alpha\cos\beta + \sin\beta\cos\alpha.$
\end{enumerate}

\emph{Gợi ý:} \href{https://vi.wikipedia.org/wiki/%C4%90%E1%BB%8Bnh_l%C3%BD_Ptoleme}{Định lý Ptoleme}

\subsection*{Vẽ đồ thị của hàm}
\emph{Chú ý}: Ta có thể vẽ đồ thị của hàm có dạng $y=Af(k(x-a))+b$ theo đồ thị của hàm $f(x)$
\begin{itemize}
    \item $y=f(x-a)$: đồ thị ban đầu được tịnh tiến theo trục $Ox$ một đại lượng $a$.
    \item $y=f(x)+b$: đồ thị ban đầu được tịnh tiến theo trục $Oy$ một đại lượng $b$.
    \item $y=Af(x)$: đồ thị xuất phát được giãn ra $A$ lần theo trục $Oy$.
    \item $y=f(kx)$: đồ thị xuất phát được giãn ra $1/k$ lần theo trục $Ox$.
\end{itemize}
\textbf{Bài 1.4:} Vẽ đồ thị các hàm số trong hai bài tập ở trên bằng

a. Desmos 
\vspace{5pt}

b. Python (đối với hàm tuần hoàn thì vẽ trong khoảng $[-\pi;\pi]$; đối với các hàm khác, lựa chọn điểm đầu và cuối sao cho thu được mọi miền của hàm)
\vspace{5pt}

\textbf{Bài 1.5}: Vẽ một hình tam/tứ/ngũ/lục giác đều bằng Desmos và Python.
\vspace{5pt}

\textbf{Bài 1.6:} Giải các phương trình sau thông qua việc vẽ đồ thị bằng Python
\begin{enumerate}[label=(\alph*)]
    \item $\tan x= x.$
    \item $\ln x = x-2.$
    \item $x^3 -15x =4.$
    \item $x^5 -4x^2 +3=0.$
\end{enumerate}
\subsection*{Hàm hợp}

\textbf{Bài 1.7:} Các hàm số trong phần là hàm hợp của những hàm nào? Hãy phân tích cụ thể thứ tự của chúng.
\vspace{5pt}

\textbf{Bài 1.8:} Nguyên lý quy nạp\newline

Cho $S_n$ là một phát biểu về số nguyên dương $n$. Giả sử rằng:
\begin{itemize}
    \item $S_1$ đúng.
    \item $S_{k+1}$ đúng khi $S_k$ đúng.
\end{itemize}
Khi đó $S_n$ đúng với tất cả các số nguyên dương $n$.\newline
Sử dụng điều này để giải các bài toán sau:
\begin{enumerate}[label=(\alph*)]
    \item Nếu $f_0(x) =x/(x+1)$ và $f_{n+1}(x)=f_0(f_n(x))$ với $n=0, 1, 2,\dots$, tìm một công thức cho $f_n(x)$.
\item Nếu $f_0(x) =x^2$ và $f_{n+1}(x)=f_0(f_n(x))$ với $n=0, 1, 2,\dots$, tìm một công thức cho $f_n(x)$.
\end{enumerate}
\subsection*{Giới hạn hàm số}
\textbf{Bài 1.9:} Chứng minh
\begin{enumerate}[label=(\alph*)]
    \item $$\lim_{x\rightarrow 0}\frac{\sin x}{x}=1.$$
    \item $$\lim_{x\rightarrow\infty}\left(1+\frac{1}{x}\right)^x =e=2,71828... .$$
    \item $\lim_{x\rightarrow 0}\frac{(1+x)^m -1}{x}=m.$
\end{enumerate}
\vspace{5pt}

\textbf{Bài 1.10:} Tính các giới hạn sau 
\begin{enumerate}[label=(\alph*)]
    \item $$\lim_{x\rightarrow 4}\frac{5x+2}{2x+3}.$$
    \item $$\lim_{x\rightarrow \infty}\frac{3x+5}{2x+7} .$$
    \item $$\lim_{x\rightarrow\infty}\frac{x^3 +2x^2 +3x+4}{4x^3 +3x^2 +2x+1}.$$
    \item $$\lim_{x\rightarrow\infty}\frac{3x^4 -2}{\sqrt{x^8+3x+4}}.$$
\end{enumerate}