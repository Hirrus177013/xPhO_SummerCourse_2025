\titleformat{\chapter}[display]
  {\normalfont\Large\bfseries}{\centering Tuần 9}{10pt}{\centering\Huge\bfseries}
  
\chapter{Nhập Môn Cơ Học Giải Tích}

\section{Liên kết động học}

\subsection{Bậc tự do}

\subsection{Liên kết Holonom và liên kết phi Holonom}

\subsection{Lực bị động trong bài toán liên kết Holonom}

\subsection{Ứng dụng đạo hàm toàn phần và ma trận Jacobian trong bài toán liên kết Holonom}

\section{Cơ học Lagrange}

\subsection{Nguyên lý tác dụng tối thiểu}

\subsection{Phương trình Lagrange loại II}

\subsection{Phương trình Lagrange loại I}

\subsection{Động lượng suy rộng}

\subsection{Định lý Noether}

% Continuous symmetry

% Bài tập ví dụ: Xác định phương trình vi phân mô tả chuyển động của con lắc kép.

\subsection{Giải phương trình chuyển động bằng phương pháp Runge-Kutta 4}

\subsection{Tính toán lực bị động dựa trên phương trình Lagrange loại 2}

\section{Các lý thuyết cơ học giải tích khác}

\subsection{Cơ học Hamilton}

\subsection{Nguyên lý Gauss về liên kết tối thiểu}

\subsection{Phương trình Appell cho cơ hệ phi Holonom}
