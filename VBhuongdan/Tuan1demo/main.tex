\documentclass[a4paper,12pt]{book}

\usepackage{lmodern}
\usepackage[utf8]{vietnam}
\usepackage{geometry}
\geometry{top=2.5cm, bottom=2.5cm, left=3cm, right=3cm}
\usepackage{titlesec}
\usepackage{setspace}

% Định dạng tiêu đề chương
\titleformat{\chapter}[display]
  {\normalfont\Large\bfseries}{\centering Tuần 1}{10pt}{\centering\Huge\bfseries}
  
\begin{document}

\chapter{Mở đầu về giải tích}


\begin{itemize}
    \item Giải tích là toán học của sự thay đổi.

    \item (lịch sử)

    \item (lịch sử)

    \item (lịch sử)
    \item (lịch sử)

\end{itemize}
\noindent Trong tuần 1, chúng tôi trình bày nội dung về hàm số và giới hạn của hàm số, đạo hàm cùng ứng dụng của chúng.
\newpage
Nội dung của tuần được chia làm: Hàm số và Giới hạn, Đạo hàm cùng Ứng dụng của đạo hàm. Trong đó, phần Hàm số và Giới hạn đã giới thiệu khái niệm về hàm số, các cách biểu diễn hàm số (đồ thị, đại số, số liệu, lời nói) cùng với khái niệm về giới hạn hàm số đi cùng các quy tắc lấy giới hạn; phần Đạo hàm giới thiệu khái niệm về đạo hàm, ý nghĩa toán học nói chung và hình học của đạo hàm của một hàm số cùng với các quy tắc lấy đạo hàm; cuối cùng, phần Ứng dụng của đạo hàm đề cập đến các khái niệm xấp xỉ tuyến tính, vi phân, cực trị hàm số...

\vspace{5mm}
Để giúp ích cho việc làm quen kiến thức đã được trình bày và cả những kiến thức quan trọng nhưng không được trình bày, phần bài tập của tuần đã được soạn sẵn được phân chia theo một thứ tự hợp lý, khuyến khích việc làm tuần tự. Cụ thể, các bài tập được phân bố như sau:
\begin{itemize}
    \item Từ... đến ..., các phép biến đổi đồ thị hàm số: tịnh tiến và kéo giãn được giới thiệu
    \item Phần Nguyên lý quy nạp đề cập đến một phương pháp chứng minh quan trọng trong toán học, bài tập ... ứng dụng trực tiếp kiến thức về hàm số và phương pháp này. Các bài tập sau đó có sự xuất hiện của phương pháp quy nạp là ...
    \item ...
\end{itemize}
Tài liệu tham khảo kiến thức cơ bản cho tuần này là
\begin{itemize}
    \item \emph{Calculus I}, Jame Stewart, chương...
    \item ...
\end{itemize}
Ngoại trừ các tài liệu trên, bạn đọc có thể đọc thêm 
\begin{itemize}
    \item Về các ứng dụng của nguyên lý quy nạp...
    \item Về lịch sử hình thành và phát triển của giải tích..
    \item Câu chuyện về lãi suất kép và hàm $e^x$...
\end{itemize}

\end{document}
